\documentclass{article}
\usepackage[utf8]{inputenc}
\usepackage[T1]{fontenc}
\usepackage{textcomp}         % Flere tegn
\usepackage{babel}            % Språk
\usepackage{amsmath, amssymb} % Bedre mattemiljø, flere symboler
\usepackage{enumitem}         % Bedre lister
\usepackage{float, graphicx}  % Figurer
\usepackage{booktabs}         % Bedre tabeller
\usepackage{import}           % Importer bilder fra Inkscape, pgf etc.
\usepackage{xifthen}          % Flere kondisjonelle kommandoer
\usepackage{pdfpages}         % Sett inn pdf som figur
\usepackage{transparent}
\usepackage{listings}
\usepackage[parfill]{parskip}
\usepackage[
  margin=1.5cm,
  includefoot,
  footskip=30pt,
]{geometry}

\definecolor{mygreen}{rgb}{0,0.6,0}
\definecolor{mygray}{rgb}{0.5,0.5,0.5}
\definecolor{mymauve}{rgb}{0.58,0,0.82}

\lstset{
  backgroundcolor=\color{white},   % choose the background color; you must add \usepackage{color} or \usepackage{xcolor}; should come as last argument
  basicstyle=\ttfamily,            % the size of the fonts that are used for the code
  breakatwhitespace=false,         % sets if automatic breaks should only happen at whitespace
  breaklines=true,                 % sets automatic line breaking
  captionpos=b,                    % sets the caption-position to bottom
  commentstyle=\color{gray},       % comment style
  deletekeywords={...},            % if you want to delete keywords from the given language
  escapeinside={\%*}{*)},          % if you want to add LaTeX within your code
  extendedchars=true,              % lets you use non-ASCII characters; for 8-bits encodings only, does not work with UTF-8
  firstnumber=0000,                % start line enumeration with line 1000
  keepspaces=true,                 % keeps spaces in text, useful for keeping indentation of code (possibly needs columns=flexible)
  keywordstyle=\color{mygreen},    % keyword style
  language=C,                      % the language of the code
  morekeywords={version},          % if you want to add more keywords to the set
  numbers=none,                    % where to put the line-numbers; possible values are (none, left, right)
  numbersep=5pt,                   % how far the line-numbers are from the code
  numberstyle=\tiny\color{mygray}, % the style that is used for the line-numbers
  rulecolor=\color{black},         % if not set, the frame-color may be changed on line-breaks within not-black text (e.g. comments (green here))
  showspaces=false,                % show spaces everywhere adding particular underscores; it overrides 'showstringspaces'
  showstringspaces=false,          % underline spaces within strings only
  showtabs=false,                  % show tabs within strings adding particular underscores
  stepnumber=1,                    % the step between two line-numbers. If it's 1, each line will be numbered
  stringstyle=\color{mymauve},     % string literal style
  tabsize=4,	                   % sets default tabsize to 2 spaces
  title=\lstname,                  % show the filename of files included with \lstinputlisting; also try caption instead of title
  frame=single,
}
\author{
    Fagervik, Jørgen\\
    Håkonsen, Iver
}
\title{Theoretical Exercise 1 -- TDT4205}

% Figurer fra Inkscape konvertert til pdf+LaTeX inluderes med \incfig{...}
\newcommand{\incfig}[2][1]{%
    \def\svgwidth{#1\columnwidth}
    \import{./figures/}{#2.pdf_tex}
}


\begin{document}
\maketitle

\subsection*{Parameter passing}

The value returned from the \texttt{main} function is 23. The arguments to
\texttt{increment\_with\_value} are \texttt{int}s, which means that the function will be called by
value, not by reference. For the function to work as the name intends, the type of the parameter
\texttt{a} instead has to be \texttt{int *}. 

\begin{lstlisting}
#include <stdio.h>

int a = 23;
void increment_with_value(int *a, int b) {
    *a += b;
}

int main(void) {
    increment_with_value(&a, 1);
    return a; // 24
}
\end{lstlisting}


\subsection*{Symbols}

\texttt{b} is a local variable inside the function \texttt{increment\_with\_value}, and will be
allocated on the stack when the function is called.

\subsection*{C arrays}
\begin{enumerate}[label=\alph*.]
    \item The call to \texttt{strcpy} copies the 14 chars stored in \texttt{t} into \texttt{s}, but
        \texttt{s} only fits 12 chars. \texttt{strcpy} does not do bounds checking at all, and
        writes 2 chars beyond, and overflows into \texttt{foo}. \texttt{foo} is an \texttt{int},
        so it at least two bytes wide, and the values \texttt{'2'} and \texttt{'3'} are written into
        it, i.e., \texttt{foo} is now \texttt{0x3332}, or 13106 in decimal. This leads to
        \texttt{foo = 13106} being printed.

    \item This problem is known as buffer overflow, which happens when a program either reads or
        writes beyond the allocated space of an array.

    \item Some patched versions of \texttt{gcc} has this option set already, and it has to be disabled with
        \texttt{-fno-stack-protector}. In stock versions of \texttt{gcc} it has to be enabled with
        \texttt{-fstack-protector}.

    \item If line 5 was replaced with \texttt{static int foo = 0;}, the problem in this particular
        case would be solved. \texttt{foo} would be allocated in static memory instead of on the
        stack, and the stack smashing would not hit it. It does not solve the underlying problem of
        stack smashing.
\end{enumerate}

\newpage
\subsection*{Functions and variables}
\begin{enumerate}[label=\alph*.]
    \item \

        \begin{tabular}{ll}
            Symbol           & Memory segment \\ \hline
            \texttt{rec()}   & text\\
            \texttt{c}       & Initialized Data Segment, read-only area \\
            \texttt{d}       & Uninitialized Data Segment (bss) \\
            \texttt{counter} & Uninitialized Data Segment (bss) \\
            \texttt{a}       & Stack\\ \hline
        \end{tabular}
    \item
        Running the program causes a segmentation fault, caused by a stack overflow.
\end{enumerate}

\end{document}
